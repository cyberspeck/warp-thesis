
\chapter{Introduction}
\label{chap:intro}
% o General: Start with very general description and focus step by step on your topic
% o keep in mind, that the introduction usually contains the most references
% o Introduction to main topic (e.g. Radiotherapy, MRI, ...) including historical review (1-2pages), Purpose of Radiotherapy ; Regardless of your actual topic, put it in context to conventional photon radiotherapy
% o Basic principles of Physics related to your topic
% o (e.g. Photo effect, Compton effect, Bethe-Bloch equation, ...)
% o Technological Background related to your topic
% o Give a general descriptions about the devices used in your thesis
% o (e.g. Linac, Afterloader, Synchrotron, Detectors...)
% o Overview of literature connected to your topic
% o Purpose of the thesis
% o based on the literature research, describe which information is missing, describe briefly what your thesis is about and what is the novelty of your work

\section{Photon - matter interactions}
\label{sec:photon}
%\todo{too simple?}

As light passes through matter, its intensity decreases.
This phenomenon is due to photons interacting with electrons, nuclei, and their electric fields.
All processes either change the direction they travel in, alter their energy, or result in the disappearance of single photons.
The probability of these interactions differ for each material (dependent on its density; proton number $Z$) and photon energy ($h\nu$). \\

If a photon's energy exceeds the binding energy of an orbital electron, the \textbf{photoelectric interaction} can occur.
Also known as 'photo effect', it describes a photon being completely absorbed by a tightly bound orbital electron which then is ejected from its atom.
The now free electron is called 'photo-electron'. Its kinetic energy is the difference of the photon's energy and the electron's binding energy:
\begin{align}
E_{kin} &= h\nu - E_{binding}
\end{align} \\

Instead of being absorbed, photons might also just 'bounce off' electrons or entire atoms, transferring momentum and, in some cases, part of their energy to the particle they collide with.
\textbf{Rayleigh (coherent) scattering} happens when a photon interacts with a tightly bound orbital electron (transferring momentum to the entire atom).
This event can be seen as elastic, because only a negligible part of the photon's energy is transferred.

The \textbf{Compton effect (incoherent scattering)} involves a essentially free electron, such as an orbital electron with a relatively small binding energy compared to the photon's energy.
Due to the weak binding, momentum is transferred only to the electron.
This 'recoil electron' (or 'Compton electron') leaves its atom with a significant kinetic energy, which originated from the scattered photon.
Since the photon loses part of its energy, the event is considered inelastic. \\

When a photon with an energy above $1.02 \, MeV$ passes through the electric field of a nucleus, it might disappear to create an electron-positron pair.
This effect is called \textbf{pair production}.
The threshold of $1.02 \, MeV$ equals exactly the rest mass $E_m = 2m_ec^2$ for the two equally heavy particles.
The new particles travel in opposite directions with the same kinetic energy:
\begin{align}
 E_{kin} = \frac{h\nu - 1.02 \, MeV}{2}
\end{align} \\

A photon with energy of the order of $2 \, MeV$ or higher can also interact directly with the nucleus.
Such a \textbf{Photonuclear reaction} is similar to the photo effect, in the sense that the photon is completely absorbed.
Its energy is transferred to the nucleus resulting in the emission of either a proton or neutron. \cite{Podgorsak, Maidment2014}\\

\subsubsection{Attenuation}
The aforementioned interactions result in a gradual decrease of light intensity as it travels through matter.
The combined effect is described by \textbf{Beer's law}:

\begin{align}
I(x) = I_0 e^{-\mu(h\nu,Z)x}
\end{align}

where $x$ is the thickness of a homogeneous material and $\mu$ its linear attenuation coefficient.
The different probabilities for the interactions to occur is implicitly considered by the attenuation coefficient $\mu(h\nu,Z)$ (see Figure \ref{fig:attenuation_water} and \ref{fig:attenuation_iron}).

For a photon being transmitted through matter with varying properties, the attenuation coefficient changes, too.
After travelling a distance $d$, the intensity can be expressed as:

\begin{align}
\label{eq:mu_int}
I(x) = I_0 e^{- \int_{0}^{d} \mu(x)dx}
\end{align}

Where $\mu(x)$ describes the attenuation at every distance $x$.

%\todo{better diagramm??}
%\todo{explain LET? (linear energy transfer)}
%\todo{explain 'dose'?}

\begin{figure}[h!]
	\centering
	\includegraphics[width=0.7\linewidth]{../fig/intro/Ironattenuation}
	\caption{Photon attenuation for iron; \cite{Materialscientist}}
% source: by 'Materialscientist', via Wikimedia Common %(\url{https://commons.wikimedia.org/wiki/File:Ironattenuation.PNG?uselang=en})\\ GFDL %(\url{http://www.gnu.org/copyleft/fdl.html})\\ CC BY-SA 3.0 (\url{http://creativecommons.org/licenses/by-sa/3.0})
	\label{fig:attenuation_iron}
\end{figure}

\section{Radiobiology}
\todo{better name for chapter! maybe more general?}
\label{sec:cell}
\subsection{The human cell}
All living creatures consist of cells working together to form what is called tissue.
A collection of tissues which perform one or more functions is considered an organ. \\

Even though different types of cells exhibit distinctive traits which set them apart, all of them originated from the same totipotent zygote containing a original set of DNA.
A zygote is a stem cell, it has the ability to replicate indefinitely, passing its DNA on to the resulting daughter cells.
At the same time, it can change into any type of body cell. This feature is why it is called 'totipotent'.
As soon as the zygote has divided into a sufficient number of identical cells, all of them differentiate into the various human tissues.
In favour of becoming more specialised cells, they lose their totipotency.
During the early stages of an embryo they are still capable of developing into a number of different cell types, but already restricted to their own tissue type; either nerve, skin, or blood \& muscle tissue.
As those cells further specialise, they limit their potential even more.
In a fully grown human body there are still stem cells present, such as bone marrow stem cells.
Other than the zygote, bone marrow can only give rise to blood cells, but not to e.g. nerve or skin cells.
A blood cell itself cannot replicate, it is considered a 'mature cell'.

The whole process follows guidelines dictated by the DNA. Every cell inherited its own personal copy of the original set.
Inevitably, mistakes happen during its replication resulting in changes to the DNA called 'mutation'.
Most of these alterations are repaired or do not lead to changes in the cells behaviour.
As the human body ages, the repair mechanism slows down and mutations accumulate.
At one point, a cell is reprogrammed to act in a unpredictable way, giving up its duties and duplicating without restraint, forming tumours.
External factors are known to influence cell behaviour and induce such 'malign' cells (carcinogenesis).
Cancer cells usually replicate more frequent than healthy cells, eventually leading to characteristic symptoms.

Different approaches have been developed to treat cancer, not all of which are suited to tackle every type of tumour.
If the tumour's location is unknown or metastases have formed already in many places, chemotherapy might be considered.
An easily accessible tumour could be removed in a surgery.
Non invasive therapies also include radiotherapy, destroying cancer cells using radiation.

Generally, early treatments have high chances of success, but tumours are often not noticed until they reached a certain stage. 
Reliable ways of diagnosing tumours are made possible by imaging techniques visualising the interior of the human body.  \cite{Baumann2017}

\subsection{Effects of radiation}
\label{sec:irradiate}

As described in \ref{sec:photon}, light transfers some of its energy to the medium it passes through.
Most interactions, such as the photo effect, incoherent (Compton) scattering, and pair production, result in free electrons.
If the electron has a sufficiently high kinetic energy, it may free additional orbital electrons from atoms in its vicinity.
The remaining ions are then left positively charged, with a single unpaired valence electron.
This type of chemical and free electrons are called 'radicals' and considered extremely reactive.
They are likely to take part in chemical events which lead to the breakage of chemical bonds.
Such processes can induce changes in DNA sequences and eventually produce biological damage.

A irradiated cell can be affected in various ways ranging from no effect to cell death.
The cell might survive containing a minor mutation.
A more fundamental mutation might lead to carcinogenesis.
Irradiated cells might also send signals to their neighbours, inducing genetic damage known as 'bystander effects'.
However, surviving cells can also react to irradiation and becoming more resistant.

Changes to the DNA might not become apparent ever, others take years until they result in biological effects.
A well known long term consequence of ionising radiation is leukaemia.
Damage to germ cells (sperm/egg) might even result in genetic damages expressed in subsequent generations.

While imaging modalities utilising x-rays are designed to apply a dose as little as possible to keep effects of irradiation low, radiotherapy makes use of the lethal effects targeting cancer cells. \cite{Podgorsak, Maidment2014}

\section{Imaging modalities}
\subsection{X-ray projection imaging}
A widely used imaging technique based on photon interactions is X-ray projection.
Its setup is made up by a light source, the object of interest, and a detector.
Since the technique is about projection, a patient needs to be placed between an X-ray tube and the detector (usually a film-cassette or digital sensor).
In the first stage of the imaging process, X-ray photons emitted by the tube enter the body.
Next, while travelling through human tissue, they interact with its atoms in various ways (see \ref{sec:photon}).
These processes govern how much radiation is absorbed or scattered.
Finally, Photons which make it through the patient are recorded as they reach the detector on the opposite side.
This results in a negative greyscale image, where brightness values correspond to the intensity reduction.
Low intensity (= high absorption) leads to bright spots on the image and vice-versa.
The whole process could also be described as 'the projection of attenuation shadows on to the detector', since the radiation absorption directly depends on the attenuation coefficient. The attenuation, on the other hand, depend on the tissue's properties (e.g. proton number Z, density, etc).
Consequently, the attenuation shadows depict inner structures of the patient.

Soft tissue such as brain matter and muscles absorb only little radiation, casting a lighter shadow (dark areas on image) than bone which absorbs more photons (bright areas).
Anything other than bone differs only slightly in attenuation, owing to the relatively small difference in atomic numbers and density.
For this reason, X-ray projection imaging is considered reliable when it comes to diagnose bone fractures, while at the same time, it is not suited to clearly delineate soft tissue structures.
 
Other imaging modalities are better suited for the latter, like medical ultrasound and Magnetic Resonance Imaging (MRI), to name a few.
They are preferred for non invasive soft tissue examinations.
If these imaging modalities are no option, X-ray projection can still be of some use in combination with contrast agents.
Such substances fill e.g. the bloodstream with heavier atoms, which can be clearly seen against the dark background of surrounding soft-tissue.
In CT angiography, for instance, iodine is administered intravenously enhancing vessel to vessel-wall contrast.
In studies of the abdomen a diluted iodine solution or barium compounds swallowed by the patient leads to improved visibility of the gastrointestinal tract.

For patients allergic to those chemicals, a number of alternative agents have been developed.
Unfortunately most introduce slight, sometimes serious side-effects.
There is ongoing research to find materials yielding enhanced contrast while at the same time minimising adverse reactions, a promising candidate being gold nanoparticles. \cite{Podgorsak, Maidment2014}


\subsection{Computer Tomography - CT}
\todo{The most important thing about CT is that it is super fast and that it does not introduce distortions. The dissadvantages are the radiation dose and low soft tissue contrast}

Computer Tomography (CT) is a three-dimensional (3-D) imaging modality based on the measurement of X-ray attenuation.
The technique has evolved from 2-D X-ray scanning.
By mounting source and detector on a rotary ring with a patient at the centre, projections from any angle can be obtained.
However, in contrast to 2-D projection methods, the detector resembles an arc made up by 800 to 900 neighbouring detector elements.
A single 'image' taken by the detector is therefore only in 1-D.
Yet, by repeating this process from a sufficient number of different angles and along the entire patient (z-axis) a 3-D model can be computed.
In contrast to 2-D methods, where the patients interior is projected/compressed onto a flat image, CT preserves the exact location information. This feature led to a radical improvement in diagnostics.	 \\

Since its clinical introduction in 1971, CT has become a widely used 3-D imaging modality for a range of applications including radiation oncology. Especially in radiation therapy, knowledge of the exact geometry is crucial, which is why CT plays such an important role in treatment planning (see \ref{sec:planning}). \cite{Podgorsak, Maidment2014}

\subsubsection{3-D image reconstruction}
As a photon passes through the patient, it encounters different materials associated with characteristic linear attenuation coefficients.
It is practical to think of the scanned body as a collection of $N = N_X\cdot N_Y\cdot N_Z$ finite size cubes ($\Delta x$ cube length) called 'voxels' (analogous to pixels in a 2-D digital photograph).
The entire model can then be regarded as a 3-D matrix, with the attenuation coefficients $\mu_i$ of the voxels as its entries.
Figure \ref{fig:voxel_matrix} represents a ($4, 4, 1$) matrix.
It depicts the path an X-ray may follow passing through voxels with different values $\mu_i$.
This discretisation allows us to change equation \ref{eq:mu_int} to:

\begin{align}
\label{eq:mu_sum}
I(x) = I_0 e^{- \sum\limits_{i=1}^{N_X} \mu_i \Delta x}
\end{align}

The initial and final intensities can be read of the settings of the X-ray tube and the detected signal.
Based on these values, image reconstruction algorithms derive the three-dimensional linear attenuation coefficient matrix.
For convenience, the computed numbers are converted to Hounsfield Units which are displayed in the final image. \cite{Podgorsak, Maidment2014}

\begin{figure}[!htb]
	\centering
	\includegraphics[width=0.5\linewidth]{intro/screenshot001.png}
	\caption{Simplified attenuation matrix (4,4,1); \cite{Maidment2014}}
	\label{fig:voxel_matrix}
\end{figure}

\subsubsection{Hounsfield Units}

In a final CT scan, voxel values are recorded in Hounsfield Units (HU), which relate to the attenuation of water at room temperature:

\begin{align}
HU_{material} = \frac{\mu_{material} - \mu_{water}}{\mu_{water}} \cdot 1000
\end{align}

Table \ref{tab:HU} lists types of human tissue and their values on the HU scale.
Generally, HU values range from -1024 to +3071 (12 bit), but the upper limit can be extended to 15,359 (14 bit) if materials with even higher attenuation need to be visualised (e.g. implants).

\begin{table}[]
	\centering
	\caption{Average HU values for various types of human tissue}
	\label{tab:HU}
	\begin{tabular}{@{}ll@{}}
		\toprule
		Substance           & HU                     \\ \midrule
		Air                 & –1000                  \\
		Lung                & –750 (–950 to –600)    \\
		Fat                 & –90 (–100 to –80)      \\
		Water               & 0                      \\
		Muscle              & +25 (+10 to +40)       \\
		Brain, white matter & +25 (+20 to +30)       \\
		Kidneys             & +30 (+20 to +40)       \\
		Brain, grey matter  & +35 (+30 to +40)       \\
		Blood               & +55 (+50 to +60)       \\
		Liver               & +60 (+50 to +70)       \\
		Compact bone        & +1000 (+300 to +2500)  \\ \bottomrule
	\end{tabular}
\end{table}

Typically, CT scans are displayed on Computer monitors, which imposes the need to map the HU values to a 8-bit greyscale (256 steps of luminosity).
Since the number of possible values (dynamic range) on the HU scale is 16 times the shades of grey on a screen (12-8 = 4 bit difference; equivalent to a factor of $2^4$) the screen cannot convey all details at the same time.
A linear mapping would result in 16 neighbouring HU values being compressed to the same brightness on the sreen.
This way, the brightest (bone) and darkest parts (soft tissue) of the image would be clearly distinguishable.
At the same time small differences (<16 HU) would appear as exactly the same intensity.
However, most of the time, the doctor's focus might lie either on soft tissue or bone material.
Bearing in mind that soft tissue values range only from 10 HU to 70 HU at most (see table \ref{tab:HU}), such a compression would make distinguishing tissues using CT very unreliable.
Instead of showing detail from the lowest to the highest value, a range of values - a so called window - can be chosen.
Let's assume, for example, a range from -100 to 155 HU to be of interest.
This selected range can be mapped directly and uncompressed to a 8-bit greyscale.
Any values above 155 HU will be assigned the brightest value (white = 255), below -100 the darkest (black = 0).
While showing very good soft tissue contrast, all bones would be depicted with exactly the same brightness (255), even though they might have a varying HU values.
For bone structures a range from 300 to 2500 HU might show sufficient contrast.
Standard computer programs used to display CT images allow the user to change the window interactively to any value range. \cite{Podgorsak, Maidment2014}

\subsubsection{Image acquisition}
The time necessary to collect 1-D attenuation projections from sufficient angles is called 'acquisition time'.
In 2-D X-ray scanning only one picture is taken, while a 3-D CT model is made up of a photo sequence.
If the patient moves during the imaging process, the final model would show motion artefacts which might lead to wrong conclusions.
Consequently, CT scanners are designed to minimise acquisition time while ensuring sufficient image quality.
Very fast CT protocols result in smaller resolution, because less images are taken.
It has to be said though that CT acquisition time is usually significantly shorter than MRI. \cite{Podgorsak, Maidment2014}

\subsubsection{Image quality} %282
%\todo{By saying this I do not refer to the same as 'soft tissue contrast' which is dependent on how much soft tissue differ in signal intensity. Instead I refer to a general trait of image quality that would apply for any imaging modality: If the SNR is too low 'low contrast' details blend in with the noise and are useless - this corresponds to a very small 'low contrast resolution'. In this case 'resolution' does not talk about pixel number but instead about distinguishable steps on the brightness scale that are not overlaid with noise.}
Additionally to the relatively short acquisition time, CT scans show little distortion compared MRI (see \ref{sec:MRI}), which is why they are often used as 'gold standards' (reference scans used for MRI calibration or distortion assessment).

Apart from distortion, there is also the 'low contrast resolution' of the scan.
This feature directly relates to how much structures and their surroundings have to differ in signal intensity to be clearly distinguishable by doctors.
This aspect of image quality is mainly limited by noise.
Noise is random patterns underlying the actual signal which always present to some extent.
It's prominence in the final image is described by the Signal to Noise Ratio (SNR).
If the SNR is too low, fine structures blend with the noise and cannot be distinguished. 
Strategies to achieve a high SNR include raising the initial photon flux (intensity) or employing contrast agents.
The intensity is governed by the tube current, which is limited by the heat capacity of the tube and health considerations regarding the patient.

Alternatively, the spatial resolution can be decreased, effectively combining neighbouring image slices.
This way the SNR for the combined slices is increased, but fine structures along the z-axis might be lost due to the reduced resolution. \cite{Podgorsak, Maidment2014}

\subsubsection{Health considerations}
CT scans describe the attenuation throughout a patient, which is directly related to how much energy is transferred from photons to matter.
Only because x-rays are absorbed by the human body, this imaging modality gives insight in the density distribution of a body's interior.
However, this transferred energy is capable of causing biological damage. (see \ref{sec:irradiate})

While the radiation dose administered during a single CT scan is almost negligible, patients receiving this dose regularly end up with a potentially harmful accumulated amount of radiation.
Cancer patients, for instance, need to be imaged frequently during treatment planning.
On the one hand, the dose is necessary for an effective treatment / removal of the tumours.
On the other hand, the long term effects of the administered dose lead to induced cancer.
However, patients might die before those consequences come into effect.
Therefore, it's typically children (who received a great number of CTs) to suffer from induced cancer occurring up to 40 years later.
So while the benefit from using CT for diagnostics far outweighs the damage, there have been major efforts to reduce dose while maintaining reasonable image quality.
\cite{Murphy2007, Brenner2001, Sodickson2009, Smith2007, McCollough2009, Goldman2013}


\subsection{Magnetic Resonance Imaging - MRI}
\label{sec:MRI}
Magnetic Resonance Imaging (MRI) is a 3-D imaging modality based on Nuclear Magnetic Resonance (NMR), a phenomenon discovered by physicist Isidor I. Rabi in 1938.
Atomic particles such as protons have an inherent quantum mechanic feature called 'spin', which is associated with a magnetic moment $\mu$.
Without an external field a proton's spin is oriented in a random direction in space and so is its magnetic moment.
The sum of magnetic moments belonging to a great number of protons results in a net magnetisation.
Due to their random orientation, the net magnetisation will be zero.
This is because, on average, for every proton's spin there is always another particle's spin oriented exactly the opposite way, cancelling its magnetic moment.

In the case of an applied external magnetic field, the spins will either align parallel (pointing in the same direction) or anti-parallel (opposite direction) to this field, minimising their energy.
Parallel protons have a lower energy than those pointing the other way.
In a collection of many spins, the number of parallel spins will therefore slightly dominate, resulting in a net magnetisation greater than zero (see figure \ref{fig:spin_align}).

\begin{figure}[h!]
\centering
\includegraphics[width=0.8\linewidth]{../fig/intro/spin_align}
\caption{The spins, initially oriented randomly in space, become aligned either parallel or antiparallel to an externally applied magnetic field $B_0$. \cite{Maidment2014}}
\label{fig:spin_align}
\end{figure}

By applying a short radio frequency pulse, the total external magnetic field changes and the magnetic moments start precessing around that new external field.
The pulse duration is usually chosen as to flip the spins by a $90^o$ angle.
They are now oriented in the transverse plane to the original external field, and so is the resulting net magnetisation.
Similar as to how a spinning top rotating at an angle to the direction of gravity precesses, the magnetic moments will now precess about the direction of the external field with a frequency linearly proportional to the external field strength.
This precession movement can be detected as induced current in a pick-up (receiver) coil, because the net magnetisation still follows the spins' orientation. 
Again, the particles would like to minimise their energy by aligning their spins to the external field, but in order to do so they need to give away the additional energy, transferring it to the surrounding lattice.
Those spin-lattice interactions happen with different efficiency depending on the tissue.
The time it takes the spins to align is expressed in a material specific time constant $T_1$.
Shortly after applying the radio frequency pulse, regions of the body where magnetic moments align quickly (short $T_1$) have a stronger net magnetisation (in the direction of the external field) than those where energy is being transferred slowly (long $T_1$).

At the same time, the spins interact with each other, affecting the local magnetic field and spins in their vicinity.
The magnetic moments, which started out precessing in phase directly after the radio frequency pulse flipped them, will precess at slightly different frequencies, due to the small fluctuations of the local magnetic field.
The differences cause the collection of magnetic moments to 'de-phase' and the net magnetisation in the transverse plane to vanish.
This process caused by spin-spin interactions is described by the material specific time constant $T_2$
Eventually, all spins will be again aligned either parallel or anti-parallel to the external field, just as they were before the RF-pulse.

Applying the RF pulse with a homogeneous field strength along the whole body would excite all spins simultaneously.
In order to localise differences in tissue magnetisation, the RF pulse is instead combined with a linear magnetic gradient field 'selecting' a slice to be imaged at a time.
The rest of the body is unaffected, and the signal measured directly after such a pulse originates only from the chosen slice.
Consequently, collecting data on the net magnetisation throughout the body in different locations ('scanning' the patient slice by slice) results in a 3-D image.
As the region of interest is usually limited to a specific organ, receiver coils are available in different sizes and shapes, designed to fit the patient with a comfortable but narrow space in between.

Doctors can choose to create images that reflect the spin-lattice interactions ($T_1$ weighted) or the spin-spin interactions ($T_2$ weighted). 
The set of parameters governing how the tissue is excited and data acquired are called 'image sequence'.
Delineating tumours or lesions is often accomplished by looking at both $T_1$ and $T_2$ weighted images and drawing the right conclusions.

Soft tissue contains a lot of water, which is made up by oxygen and hydrogen.
Hydrogen nuclei are single protons and their NMR is what MRI is tuned to make visible.
This is why soft tissue appears as bright areas in MRI, whereas bone material has only little contrast. \cite{Currie2013}

Studies have shown that delineating certain types of tumours, for example prostate cancer, is more accurate using MRI images than using CT. \cite{Rasch1999, Debois1999a, Roach1996}
Whereas in diagnostics, MRI scans might be sufficient, radio therapy treatment planning needs more/other information about the regions surrounding the tumour.
MRI lacks some of those insights, which is why CT is still an essential part for external beam radiation therapy. (see \ref{sec:planning})

\subsection{Non morphologic Imaging}


% https://www.youtube.com/watch?v=J_aamnpRJE8
% mri - from picture to proton S. 331 (im pdf auf 344)
Diffusion weighted (DW) imaging quantifies molecular diffusion in the body.
Additionally to the gradients needed to select a slice (strenght $~3-5\, mT/m$, duration $2-4\, ms$), the sequence for DW imaging applies two strong and long consecutive and opposite gradients (strenght $~30-50\, mT/m$, duration $20\, ms$) during which molecules may move due to diffusion.
Molecules which are restricted in their movement will experience two equally strong but opposite magnetic fields.
The first will cause them to precess with a certain speed (linear with field strength), effectively changing their phase.
The second will cause them to precess exactly the other way around (same strength, but other direction), effectively returning them to their initial state.
After those two gradients (which actually have to be applied in all three Cartesian directions), the remaining signal is measured by the receiver coil.
Those molecules which are free to move, however, will not experience a constant field strength.
As they move through the body they will precess at varying speeds during the first and again different speeds during the second gradient.
As a result, they will be out of phase when the signal is measured by the receiver coil.
DW imaging can be used to diagnose acute strokes (brain infarct), because areas with restricted diffusion (blocked blood flow) show a strong signal compared to healthy tissue with normal diffusion in the resulting image.
Since the time necessary to allow the molecules to move during the two gradients is relatively long, the image will naturally be $T_2$ weighted.
This is taken into account by taking a second image which is also $T_2$ weighted, but does not apply two consecutive opposite gradients.
The difference between the DW and the not DW weighted sequences reflects the actual contribution of diffusion (apparent diffusion
coefficient - ADC).

\todo{Perfusion}
\todo{Spectroscopy}
\todo{fMRI, rewrite}

A more recent usage of MRI based on the BOLD (Blood Oxygenation Level Dependent) effect enables us to 'watch the brain thinking'.
Nevertheless, high field scanners are key to developing new methods such as functional MRI (fMRI) of the brain \cite{Duyn2012} and observing ``metabolic reactions occurring in a human body in addition to producing very precise images of body structures'' \cite{Wada2010}.

\subsubsection{Image Quality}
Contrary to CT, MRI is prone to distortion, due to field inhomogeneities.
For most applications, small position shifts and deformations are of minor importance.
In RTTP however, those effects can have a big impact.
Therefore, MRI scanners usually come equipped with an internal distortion correction algorithm.
Those methods are developed by the company designing the scanners.
Knowing the technical details enables them to write tailor-fit scripts which drastically reduce the distortion.

While its soft tissue contrast is superior to CT, a relatively long acquisition time is necessary to achieve a sufficient high SNR.
This leads to the risk of motion artefacts (patients moving during the scanning procedure).
To tackle this issue, resolution can be reduced, effectively combining signal from several voxels to create a single voxel, reducing the overall noise.
The trade-off is that fine structures might get lost.

The key difference between CT and MRI considered in this work is the unavoidable distortion occurring in MRI images.
Mainly caused by inhomogeneities of the magnetic fields, organs might appear shifted, elongated or shrunk.
The effect is most prominent along the outer edges of the scanners field-of-view (FOW).
In the isocentre (middle) of the scanner, the distortion is smaller, because here the field is least aberrant.


%Field of view (FOV) of the MRI scanner is smaller than the CT scanner's.


\subsubsection{Health considerations}

Strong static magnetic fields up to $3T$ are present around MRI scanners at all times, and precautions are necessary to ensure safety for patients and medical personnel.
Ferromagnetic materials (such as steel and iron) can become dangerous projectiles in vicinity of the MRI scanner.
They must be excluded from the room housing the magnet without exception.
It is also important to remember that medical implants, including but not restricted to cardiac pacemakers and hearing-aids, might malfunction or become damaged in strong fields regardless whether they contain ferromagnetic materials.
Patients with such implants might still be imaged with scanners utilising weak fields up to $0.5T$.

The RF pulses repeatedly put spins in excited states, transferring energy to the human body. MRI scanners are designed to limit the rise of a patient's body temperature to $0.5^oC$ during standard imaging. Only with either medical or appropriate psychological monitoring the limit raises to $1^oC$. For any higher values a ethics committee approval is necessary.
In general, patients should be exposed to RF fields only as strong as their thermoregulatory system is capable to cope with.

Finally, magnetic field gradients are applied together with the RF pulse.
They are switched at high frequencies leading to induced currents in conducting body tissue.
In principle, those currents stimulate nerves which might result in muscle twitching or pain.
However, gradient levels are set to avoid stimulation.
During studies some subtle biological effects have been reported, but there was no evidence pointing towards harm caused by short term exposures.
At the same time, patients suffering from epilepsy might show increased sensibility to induced electric fields in the cortex.
Such subjects should be imaged with caution. \cite{Maidment2014}


\subsubsection{Open bore MRI scanners}
\todo{low fields -> less distortion??}
% advantages of low tesla, brachytherapy
The radiation oncology department of the Vienna General Hospital (AKH) is equipped with an $0.35  T$ open-bore, c-arm MRI scanner.
This open design has proven to drastically improve the well-being of patients who experience anxiety in closed-bore scanners which is why the number of incomplete MR examinations due to a claustrophobic events is low. \cite{Enders2011a, Bangard2007}
Besides, patients who wouldn't fit in closed designed scanners can be imaged.
Furthermore, brachytherapy patients can be placed in the scanner with applicators attached.

This scanner's field is weaker than the field of a conventional closed bore scanner (1-3 Tesla).
High field strengths would result in greater resolution, better Signal to Noise (SNR) ratio, and faster imaging time.
Generally, diagnostics benefit from greater image quality.
However, at some point diagnostic accuracy stops increasing with field strength.
At the same time astonishing improvements can be achieved at low fields.
A ``combination of field independent polarisation [...] with frequency optimized MRI detection coils [...] results in low-field MRI sensitivity approaching and even rivalling that of high-field MRI.'' \cite{Coffey2013}

Apart from the often satisfactory image quality, there are considerable cost advantages to the use of lower field MRI.
The initial purchase price and the ongoing maintenance expenses are considerably lower than those of high field scanners which often use superconducting magnets cooled with liquid helium. \cite{Rutt1996}
Permanent magnets might be weaker, but do not require constant cooling.
Low fields allow facilities to build smaller rooms and magnetic objects are less dangerous.



\section{External Beam Radiation Therapy}
\label{sec:planning}
External Beam Radiation Therapy (EBRT) utilizes ionizing radiation to damage cancer cells in order to stop them from multiplying.
This prevents the growth of tumours and hopefully cures the patient. 
In conventional EBRT, photons (x-rays) in the range of 4MeV to 20MeV are used to administer the necessary dose at the location of the tumour.
Unfortunately, light interacts with all cells it is passing through until it is fully absorbed.
It releases its energy along its entire path while travelling through the patient.
This may result in energy being transferred to cells all the way from the point of entry to the point where the (weakened) ray leaves the patient.

Charged particles (e.g. protons or carbon ions) minimise the damage done to healthy tissue due to their distinctive behaviour in energy loss called ``Bragg Peak''.
They release most of their energy only shortly before being stopped completely.
\cite{Nakamura2010} This effect can be used to spare tissue lying behind the tumour from radiation entirely and also reduce the amount of energy transferred to those lying before. \cite{Paganetti2005}
A comparison betweenthe behaviour of x-rays and protons is shown in figure \ref{fig:bragg}.

\begin{figure}[!h]
	\centering
	\includegraphics[width=0.6\textwidth]{Dose_Depth_Curves.png}
	\caption{energy release of ionising radiation; \cite{Cepheiden}}
% (By Cepheiden, via Wikimedia Common;\\ GFDL \url{http://www.gnu.org/copyleft/fdl.html})
	\label{fig:bragg}
\end{figure}

While travelling through matter both types of radiation release energy mostly due to coulomb interactions with the outer shell electrons of atoms.
Knowing the electron density of the targeted tissue area is therefore essential.
In order to reach a specific penetration depth, the particles' initial energy has to be chosen accordingly.

\clearpage
\subsection{Role of CT}
Until recently, radiotherapy treatment planning (RTTP) relied heavily on Computer Tomography (CT).
There are two main reasons for this:

Firstly, CT uses low energy x-rays to create a 3D image of the patient.
The luminosity value (brightness) assigned to each voxel (like pixel, but three-dimensional) corresponds to the local radiodensity recorded in Hounsfield units ($HU$).
Materials with a higher radiodensity (e.g. bones) absorb more x-ray photons than those with less (e.g. water, brain-matter).
Calculating the electron density using data obtained with CT is an easy task and used widely for RTTP. \cite{Constantinou2012, Schneider1996}
In order not to induce new cancer cells in healthy tissue during EBRT, the radiation beams are carefully targeted using the measured radiodensity. 
This way the absorbed dose accumulates in the cancer regions, while the nearby healthy tissue receives less radiation.

Secondly, CT images generate 3D images with little distortion. Exact geometries are needed for correct RTTP.
Distorted images might lead to wrong calculations of how much energy is needed for the radiation to accumulate exactly at the target region.
If, for example, bone structure is depicted as thicker than it really is, RTTP would suggest a treatment which would deposit more energy behind the tumour than intended. The opposite holds for cases where tissue appears to be thinner, which would result in areas lying before the tumour being irradiated.
This is why CT images are preferred as they show little to no distortion.

\vspace{4cm}
\textit{Image of RTTP}
\vspace{2cm}

\subsection{Role of MRI}
Today RTTP often combines CT images with data acquired using Magnetic Resonance Imaging (MRI).
MRI scans also record luminosity values, but they do not correspond to $HU$ (which is directly related to radiodensity, measured by CT).
The signal intensity depends on other factors (see \ref{sec:MRI}) and even varies between MRI scanners.
Due to the better visibility of tumours on MRI images, RTTP often uses combined data from both imaging modalities.
However, there are some difficulties arising from combining CT and MRI for EBRT:
In order to profit from separately acquired data, the resulting images must be aligned either manually or automatically. This is a hard task since non-rigid objects (organs) change their shape and location between measurements which may lead to inaccuracies.
Therefore MRI-only radiation therapy protocols are being developed:
MRI data is used to create a Pseudo-CT, which contains information about electron density. Comparisons to using CT and MRI have shown acceptable deviations for X-ray therapy.
In charged particle therapy the resulting dose gain in healthy tissue and dose loss in cancer regions owed to inaccurately assigned electron density values is bigger.
However, current development is promising. \cite{Rank2013, Stanescu2006, Nyholm2015, Greer2015, Chen2004}



\section{Aim of this work}
The idea of only using MRI for treatment planning seems close to realisation, but there are some issues that need to be addressed first.
Due to the possible image distortion, great care needs to be taken and the images must be verified before they are used for RT target definition.
The available MRI scanner at the AKH is equipped with an on board correction algorithm which is supposed to reduce distortion.
The goal of this work is to commence the development of a quality assurance tool to asses the spatial distortion of the MRI scans (after applying the internal correction).
This is achieved by comparing MRI images to CT images used as a gold standard.
An already existing custom designed phantom is provided by the AKH Vienna for this purpose.
However, the liquid to fill the rods with has not been chosen yet.
Therefore, this work focuses mainly on the acquired data and which liquids to use the phantom with, not its entire design.
However, possible fillings have to be produced and tested.
Similar approaches are being used for distortion correction by other facilities. \cite{Price2015, Petersch2004, Torfeh2015, Wang2004, Wang2004b, Mizowaki2000}
\todo{more spectacular!}

%to me \footnote{Auszug aus \citetitle{BohemRhap}~\cite{BohemRhap} von \citeauthor{Queen}~\cite{Queen} }\\
%\section{Farrokh Bulsara aka. Freddie Mercury}
%Farrokh Bulsara war ein Ausnahmetalent schuf zusammen mit der Band Queen einige der größten Hits aller Zeiten. Noch heute ist er ein wichtiges Thema in unterschiedlichsten Medien, wie in Abb. \ref{fig:freddiehg} zu sehen ist. %
%Weitere Zitate sind in Anhang \ref{appendix:zitate} zu finden.


\newpage