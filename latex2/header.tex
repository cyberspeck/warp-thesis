\usepackage{cmap}
\usepackage[utf8]{inputenc} % Richtiges anzeigen von Umlauten und quasi allen anderen Schriftzeichen
\usepackage[T1]{fontenc} % Wichtig für alles was mehr als ASCII verwendet
\usepackage{csquotes} % Schöne Anführungsstriche mit \enquote{Text}
\usepackage{amsmath} % Bessere und schönere mathematische Formeln
\usepackage{mathtools} % Noch schönerere mathematische Formeln
\usepackage{amstext} % \text{} Macro in mathematischen Formeln
\usepackage{amsfonts} % Erweiterte Zeichensätze für mathematische Formeln
\usepackage{amssymb} % Spezielle mathematische Symbole.
\usepackage{array} % Matrizen in mathematischen Formeln
\usepackage{textcomp} % Für textmu und textohm etc. um im Fließtext keine Mathematik 
%\usepackage{textalpha} % Damit können griechische Zeichen direkt im Text verwendet werden (siehe zeichen.txt)
%\usepackage{paralist} % Für compactitem und compactenum
\usepackage{xstring} % Für IF in Titelseite
\usepackage{url}

\usepackage{tikz}
\usetikzlibrary{shapes,arrows}

%\usepackage[version=3]{mhchem} % Für Chemische Formeln
%\usepackage{braket} % Für das quantenmechanische Bra-Ket

\usepackage{geometry} % Seitenränder und Seiteneigenschaften setzen
%\usepackage[showframe]{geometry} % Anzeigen der Seitenränder, nützlich für debugging. http://ctan.org/pkg/geometry

\usepackage[bottom]{footmisc} % Zwingt Fußnoten an das Ende der Seite
\usepackage[pdftex]{hyperref} % Links richtig anzeigen. Sowohl innerhalb des Dokuments (Fußzeilen, Formeln), als auch ins Internet

\usepackage[ % Biblatex für die Zitate und Referenzen
%	maxbibnames=99,
	backend=biber,
	style=numeric-comp,
	hyperref=true,
	sorting=none,
	sortcites,
	style=ieee,
	]
	{biblatex}
%\usepackage{cite} %Biblatex alt - alternative vong valle valle valle czamler

\usepackage{xkeyval} % Erlaubt "Variablen" zu definieren, wird für Titelseite gebraucht
\usepackage{graphicx} % Wichtig für das Einbinden von Grafiken
\usepackage{caption}
\usepackage{subcaption} % Einbinden von mehreren Grafiken in einer figure


%\usepackage{dirtree} % Erlaubt das erstellen von Dateibäumen
% \dirtreecomment{Text} erstellt einen Kommentar zu dem Verzeichnis bzw. der Datei
\newcommand{\dirtreecomment}[1]{\dotfill{} \begin{minipage}[t]{0.5\textwidth}#1\end{minipage}}

\usepackage{fancyvrb} % Mehr Optionen für Verbatim
\usepackage{listings} % Zur Darstellung von Programmcode
\usepackage{pdflscape} % Querformat Seiten

\usepackage[german,british]{babel}

\usepackage{color} % Farben für den todo Befehl
\newcommand{\todo}[1]{{\color{blue}(TODO: #1)}} % Einfach \todo{Text} verwenden! Cerulean
\newcommand{\blankpage}{ \newpage \thispagestyle{empty} \mbox{} \newpage }

%Joschis Addons
\usepackage{pgfplots} % für diagramme
\pgfplotsset{compat=1.9} % more recent version, not backwards compatible?
\usepackage[section]{placeins} %Ermöglicht den Befehl FloatBarrier; bei \section{} automatisch schon dabei
\usepackage{colortbl} %Für hintergrundfarben bei Tabellen Ermöglicht \rowcolor command
%\usepackage[compact]{titlesec} %Whitespace rund um überschriften verringern

%Davids Addons
\graphicspath{ {../fig/} }
\usepackage{booktabs}
\usepackage{setspace} % http://texblog.org/2011/09/30/quick-note-on-line-spacing/
%\usepackage[colorlinks=false]{hyperref} %For creating hyperlinks in cross references
\usepackage{enumitem}
\usepackage{pdfpages} % https://texblog.org/2011/10/26/including-pages-from-pdf-documents/
\setlist{nosep}