%This kind of comments need to go to the dicussion. As mentioned above here you report on the results. So the numbers, tables and images. You can write the the rod X was better vissible than Y. One iquid waseasier to fill than other. But what you thing is more useful and why (considering all pros and cons) you need to put in the dicussion. That is why it is called discussion :)



\chapter{Discussion}
% o Interpretation of results, putting them in context with literature
% o Discuss only results which were presented in the results section and do not repeat them one by one
% o Which impacts have your results on the scientific community?
% o provide an assessment of the weaknesses and strengths of your work
% o At least 5 pages

\section{Phantom design}
    
To measure distortion the scanners must image a rigid object with known dimensions.
Such phantoms are commercially available, but often expensive and designed for a specific calibration protocol.
Some institutions build their own to fulfil exactly the requirements of a given application.
The scanner used at the AKH is a relatively rare model, which is why there are only few available off-the-shelf phantoms that would fit its coil.

For the AKH it was important to create a lightweight phantom which can be imaged by CT and MRI scanners.
Due to the underlying physics, plastics are not visible in MRI scans. CT however, also shows plastic.
See figure \ref{fig:sagittal_comparison} for a comparison (MRI/CT visibility).
Therefore, they decided on plastic rods with a suitable fluid filling.
Such a liquid should be easily produced, non-toxic, and yielding sufficient signal in MRI scans.

Commercially available phantoms often resemble water filled tanks containing plastic grids as a reference.
This design results in stronger signal, but exceeds practical weight.
There are few brands offering solutions utilising liquid fiducial markers in the shape of pellets.
They are arranged in a regular pattern surrounded by air or plastic.
The AKH's design however relies on replaceable rods, which makes it a novelty.


\section{Tested solutions}
\newpage

For measuring the position of the rods in the CT scans, the plastic rods without filling would be enough already.
Hollow plastic rods would not be visible on the MRI scans, though.
That is why the visibility of the liquids on CT is not important at all.
From now on 'signal strenght' or 'visibility' will refer to MRI scans only (see table \ref{tab:visibility}).

The selection of solutions tested were chosen for a number of reasons:
\begin{enumerate}[label=\textbf{\arabic*.}]
\item Generally, imaging techniques aims for a high signal-to-noise ratio. Therefore, Liquids resulting in brighter pixels are favoured.
\item The amount of gas in the rods should be minimised.
\item If air bubbles form, tilting the entire phantom slightly should be enough to move them to one side. The FOW of the MRI scanner is too small to show the entire phantom anyway.
\item Most tested fluids are based on water, because this makes them easy to empty and clean.
They could then be filled again with a different liquid if needed.
\item Preferably, the components which are chosen to be used for the entire phantom should be either non-toxic or harmless if not swallowed.
\end{enumerate}


To achieve high signals, liquids \textbf{\#2} to \textbf{\#10} and \textbf{\#14} to \textbf{\#15} are based on distilled water with $NaCl$ (concentration of $0.36 \, g/L$) and $CuSO_4\cdot5H_2O$ (concentration of $1.96 \, g/L$ as suggested by AAPM MR Subcommittee \cite{Jackson2009} apart vom \textbf{\#3} and \textbf{\#4} which contain double and ten-fold the concentration).
\todo{toxicity of $CuSO_4\cdot5H_2O$?}

Unfortunately, dissolved gases leave the liquid and water evaporates over time.

To improve the mobility of trapped air bubbles, generic washing-up soap was added to \textbf{\#5}, \textbf{\#6}, and \textbf{\#7}.
The higher concentrations of soap were tested as reference, but the smallest amount of 1 g/L was already enough to result in sufficient mobility.
An even lower concentration might also be acceptable.
Interestingly, the rod filled with this solution contained the least amount of gas after 2 months.
If the liquid should happen to leak from the rod, the relatively low concentration of soap would not add to its toxicity.
The visibility recorded was among the higher candidates, too.
For those reasons, and because the liquid is cheap and easy to produce, it appears to be a promising candidate.

Liquids \textbf{\#8} ($0.36 g/L$), \textbf{\#9} ($3.6 g/L$), and \textbf{\#10} (\$36 g/L\$) contain ascorbic acid.
Adding this is supposed to reduce forming of air bubbles by binding dissolved oxygen and eventually degrade to dehydro-ascorbic acid and water.
The amount suggested by \cite{Abtahi2008, Bodannes1979} is $0.00204 \; mol/L$ which corresponds to approx. ($0.36 g/L$).
As noted in \ref{sec:sol-mech}, the rods containing these liquids still contained air after 2 months.
The reason why the rod containing the highest concentration of ascorbic acid showed an unexpected colouring is not known.
Whether this is due to a contamination or a unforeseen chemical reaction involving the high concentration of ascorbic acid is unclear.
Adding ascorbic acid seems to yield no long term benefits.

Primovist (\textbf{\#11} to \textbf{\#13}) is a common contrast agent used for MRI scans \cite{VanBeers2012, Rohrer, primovist} intended to yield a strong signal.
The limited mobility of air bubbles forming makes it a unsuited candidate.

In an attempt to limit the forming of gas, \textit{agar} was used in soltutions \textbf{\#14} and \textbf{\#15}.
Agar and agarose are commonly used as basic reference material for MRI phantoms \cite{BuccioliniCiraolo1989, Mathur-DeVre1985}
Since the produced gel cannot be removed from the rods as easily as liquid candidates, and forming air bubbles were also not moving, agar is not suited for this phantom.

As an alternative 2 oils were proposed.
Since air is not soluble in oil, once a rod is completely with oil, no air bubbles should form.
Using vegetable oil would be a non-toxic solution.
This has been ruled out as a filling from the beginning, because it would eventually rot.
Mineral oil on the other hand does not rot, but it is not as easily removed from a rod as a water based liquid.
At the same time, it might not be necessary to ever replace the oil.
Once filled, the rods could be used until the surrounding plastic breaks or starts leaking.
In that case, however, it might be dangerous if not handled properly as it might leave traces on the MRI scanner bed.
As the generic motor (\textbf{\#16}) oil resulted in the highest signal intensity of all candidates it seems to be a good alternative to water based liquids.
The silicon oil (\textbf{\#17}) on, the other side, had a low signal compared to most canditates and is therefore not suited.

In conclusion, a water based liquid seems to be the right choice to start with.
Provided the amount of liquid remains high enough to fill the entire FOW of the MRI scanner.
Oil generally shows good visibility in CT and MRI scans and produces no air bubbles after closing.
Since topping up over 300 rods regularly is too time consuming, the generic motor oil seems to be a good alternative.

\section{Sequence}
\todo{why T1 weighted sequence? why this particular sequence? Influence on signal of water/oil, components.}

\section{Distortion}
%Gnerally this graphs are OK, but we discused before you left that there is need to comment of what we actually see on them, as you explained that to me. Also we dicussed that you could add a figure showing a particular slice on which a certain problem occures (why is the extra-high distortion dettected: a bubble or artefact). 

We also talked that the XY axes have to properly called, so on X distance from the middle insted of the slice etc. Based on this descriptoion we could write smth clever in the discussion.  

An air bubble might lead to a incorrect COM. Without looking at the dice-coefficient it's hard to tell why this distortion only appears to be present in a few slices.
If both indicators show unexpected local irregularities, a conclusion might be easier to draw.

\todo{discuss what is visible on the graphs}
\todo{interpret data table spit out by script}
