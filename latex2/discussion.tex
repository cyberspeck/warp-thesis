
\chapter{Discussion}
% o Interpretation of results, putting them in context with literature
% o Discuss only results which were presented in the results section and do not repeat them one by one
% o Which impacts have your results on the scientific community?
% o provide an assessment of the weaknesses and strengths of your work
% o At least 5 pages


\section{Tested solutions}

For measuring the position of the rods in the CT scans the plastic rods without filling would be enough already, they would not be visible on the MRI scans, though.

Oil generally shows good visibility in CT and MRI scanns and produces no air bubbles after closing.
However, filling all rods of the phantom with Oil is considered the last option. Water based liquids prove to be easier to clean.
Since topping up over 300 rods regularily is too time consuming, oil seems to be a good alternative.
However, if air bubbles could be easily shifted towards one end of the rod by tilting it slightly, they would lie outside the MRI scanner's field of view.
Consequently, it would be sufficient to move air bubbles to one end of the rod before imaging.


\section{Distortion}
An air bubble might lead to a incorrect COM. Without looking at the dice-coefficient it's hard to tell why this distortion only appears to be present in a few slices.
If both indicators show unexpected local irregularities, a conclusion might be easier to draw.
