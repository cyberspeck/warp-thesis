%This kind of comments need to go to the dicussion. As mentioned above here you report on the results. So the numbers, tables and images. You can write the the rod X was better vissible than Y. One iquid waseasier to fill than other. But what you thing is more useful and why (considering all pros and cons) you need to put in the dicussion. That is why it is called discussion :)



\chapter{Discussion}
% o Interpretation of results, putting them in context with literature
% o Discuss only results which were presented in the results section and do not repeat them one by one
% o Which impacts have your results on the scientific community?
% o provide an assessment of the weaknesses and strengths of your work
% o At least 5 pages

\section{Phantom design}

To be able to assess the spacial distortion of an MRI scan, a rigid object with known dimensions needs to be scanned so the cross referencing to the ground truth can be performed.
Such phantoms are commercially available, but often expensive and designed for a specific calibration protocol.
Some institutions build their own to fulfil exactly the requirements of a given application.
The scanner used at the AKH is a relatively rare model, which is why no off-the-shelf phantom that would fit its coil is available.

A previously used phantom did not fill the whole FOV, while at the same time weighing about 45kg.
This was due to the fact that it resembled a cuboid water tank with a thin plastic rod construction inside.
As the peripheral zones of the FOV are those where distortion is most pronounced, a bigger phantom was needed to assess those regions.
The new design deals reaches the outer regions and weighs much less at the same time.\\

Due to the underlying physics, plastics are not visible in MR scans, but CT scans visualize them. 
See figure \ref{fig:sagittal_comparison} for a comparison (MRI/CT visibility).
Therefore, it was decided to use plastic rods with a suitable fluid filling.
Such a liquid should be easily produced, non-toxic and yielding sufficient signal in MRI scans.

Commercially available phantoms often resemble water filled tanks containing plastic grids as a reference.
This design results in stronger signal, but exceeds practical weight.
There are few brands offering solutions utilising liquid fiducial markers in the shape of pellets.
They are arranged in a regular pattern surrounded by air or plastic.
The AKH's design however relies on replaceable rods, which makes it a novelty.

\subsection{Observed issues}

Interestingly, all water based solutions seemed to have evaporated partly.
As the rod filled with liquid \textbf{\#15} has dried starting at the end with the plastic stopper, it seems likely that, at least in this particular case, the plastic stopper did not effectively close the rod.
It might also be that the rod itself does not prevent volatile liquids from escaping slowly.
This was tested in a small experiment where an empty, closed rod was placed underwater.
After some time, water bubbles formed along its wall.
An airtight container might have led to other conclusions regarding the formation of air bubbles.
It is hard to tell if they were caused by evaporation only or if dissolved gases played a role, too.
Whatever the reasons are, the use of water based liquids seems to be suboptimal.
Despite this, the observed behaviour will still be discussed as a future airtight phantom design might benefit from drawn conclusions.

\section{Tested solutions}

For measuring the position of the rods in the CT scans, the plastic rods without filling would be enough already.
That is why the visibility of the liquids on CT is not important at all.
Hollow plastic rods would not be visible on the MRI scans, though.
From now on 'signal (strength)' or 'visibility' will refer to MRI scans only (see table \ref{tab:visibility}).

\subsection{Thoughts about choosing possible candidates}

The tested solutions were chosen for a number of reasons:
\begin{itemize}
\item Generally, imaging techniques aims for a high signal-to-noise ratio. Therefore, Liquids resulting in brighter pixels are favoured.
\item The amount of gas in the rods should be minimised.
\item If air bubbles form, tilting the entire phantom slightly should be enough to move them to one side. The FOV of the MRI scanner is too small to show the entire phantom anyway.
\item Most tested fluids are based on water, because this makes them easy to empty and clean.
They could then be filled again with a different liquid if needed.
\item Preferably, the components which are chosen to be used for the entire phantom should be non-toxic.
\end{itemize}

\vspace{1cm}

Rod \textbf{\#1} was filled with plain distilled water and intended to be used only as a reference.
It was clear from the beginning that it would not result in high signal and was never considered a possible filling.

\subsubsection{Aiming for high SNR}
To achieve a better SNR, liquids \textbf{\#2} to \textbf{\#10} and \textbf{\#14} to \textbf{\#15} are based on a solution of sodium chloride ($NaCl$ concentration of $0.36 \, g/L$) and copper(II) sulfate pentahydrate ($CuSO_4\cdot5H_2O$ concentration of $1.96 \, g/L$ as suggested by AAPM MR Subcommittee \cite{Jackson2009};  \textbf{\#3} and \textbf{\#4} contain double and ten-fold the concentration) in distilled water.
Most of these liquids resulted in an about 5 times brighter signal than plain distilled water.
Regarding the toxicity of $CuSO_4\cdot5H_2O$, the minimum dose to have caused acute toxic effects in humans is reported to be $11 mg/Kg$.
%As the concentration of $CuSO_4\cdot5H_2O$ is only $1.96 \, g/L$, even a young patient of 4 years and $15kg$ would need to drink more than 80mL to reach that threshold.
%Also, ingestion usually leads to vomiting triggered by its irritating effect on the gastrointestinal tract, which would stop further (more severe) toxic effects from happening.
%Keeping the scanner bed clean and checking the phantom for leakage before and after use should be sufficient to prevent an intoxication.
% http://pmep.cce.cornell.edu/profiles/extoxnet/carbaryl-dicrotophos/copper-sulfate-ext.html

\vspace{1cm}

Primovist (\textbf{\#11} to \textbf{\#13}) is a common contrast agent used for MRI scans \cite{VanBeers2012, Rohrer, primovist} intended to yield an even stronger signal than $CuSO_4\cdot5H_2O$ based liquids.
The major drawback is its tendency to separate from the water and stick to the container's wall.
This results in low signal at the centre and high signal along the wall, which would not necessarily pose a problem, but as the liquid forms bubbles, it is not guaranteed that the walls would be covered homogeneously.
The uneven distribution might result in wrong calculations, especially if the software tool is not programmed to cope with this behaviour.
At the same time the removal would be hardly possible.

\subsubsection{Handling dissolved gas}
Unfortunately, dissolved gases may eventually leave the liquid and form air bubbles trapped in the rod.
To improve the mobility of trapped air bubbles, generic washing-up soap was added (\textbf{\#5}, \textbf{\#6}, and \textbf{\#7}; suggestion by Data Spectrum Corporation \cite{bubbles}).
%The smallest amount of 1 g/L was already enough to result in sufficient mobility and an even lower concentration might also be acceptable.
The higher concentrations of soap were tested as reference.
%Interestingly, the rod filled with this solution contained the least amount of gas after 2 months.
If the liquid should happen to leak from the rod, the relatively low concentration of soap would not add to its toxicity.
%The visibility recorded was among the higher candidates, too.
For those reasons, and because the liquid is cheap and easy to produce, it appears to be a promising candidate.

\vspace{1cm}

Liquids \textbf{\#8} ($0.36 g/L$), \textbf{\#9} ($3.6 g/L$) and \textbf{\#10} ($36 g/L$) contain ascorbic acid.
Adding this was supposed to reduce forming of air bubbles by binding dissolved oxygen and eventually degrade to dehydro-ascorbic acid and water.
The amount suggested by \cite{Abtahi2008, Bodannes1979} is $0.00204 \; mol/L$ which corresponds to approx. ($0.36 g/L$).

\vspace{1cm}

In an attempt to limit the forming of gas, \textit{agar} was used in solutions \textbf{\#14} and \textbf{\#15}.
Agar and agarose are commonly used as basic reference material for MRI phantoms \cite{BuccioliniCiraolo1989, Mathur-DeVre1985}
%Since the produced gel cannot be removed from the rods as easily as liquid candidates, and forming air bubbles were also not moving, agar is not suited for this phantom.

\subsubsection{Non water-based liquids}
As an alternative 2 oils were proposed.
Since oil is neither soluble in air, nor able to evaporate, a rod completely filled with oil should stay free from air bubbles.
Yet, oil is not as easily removed from a rod as a water based liquid.
At the same time, it might not be necessary to ever replace the oil.
Once filled, the rods could be used until the surrounding plastic breaks or starts leaking.
Using vegetable oil would be a non-toxic solution, but has been ruled out as a filling from the beginning, because it would eventually rot.
Mineral oil on the other hand does not rot, however, it might be toxic if consumed.
%As the generic motor (\textbf{\#16}) oil resulted in the highest signal intensity of all candidates it seems to be a good alternative to water based liquids.
%The silicon oil (\textbf{\#17}) on, the other side, had a low signal compared to most candidates and is therefore not suited.


\subsection{Choosing a promising candidate}
As all rods containing water continued to lose liquid due to evaporation, only the early forming of air bubbles might indicate whether solutions effectively hinder dissolved gases to result in trapped air bubbles.
Apart from the solutions containing ascorbic acid (\textbf{\#8} ($0.36 g/L$), \textbf{\#9} ($3.6 g/L$)) all water based liquids produced some air bubbles after at least two days (see table \ref{sec:sol-mech}).
Considering the low visibility of \textbf{\#9}, the only suitable water based liquid capable of staying free from air bubbles might be \textbf{\#8}.
At the same time, long-term observations performed in an airtight container might have shown that even \textbf{\#8} only delays the process.
In the case of the used rods, such a conclusion cannot be drawn with certainty.

The rod containing the highest concentration of ascorbic acid showed a yellow colouring, caused by dehydroascorbic acid which is the result of a oxygenation process.
However, as the high concentration in \textbf{\#9} and \textbf{\#10} led to a radical reduction in signal with the tested MRI sequence, this solutions is not considered a suitable filling anyway.

Primovist (\textbf{\#11} to \textbf{\#13}) lead to a good signal, but its limited mobility of air bubbles; its tendency to accumulate along the wall; and the difficulty of cleaning the rods rule it out as a candidate.

If the forming of  a small amount of gas is not considered a problem, adding soap appears to be a reasonable solution.
The smallest tested amount of soap (\textbf{\#5}, 1 g/L) was already enough to result in sufficient mobility of air bubbles, and an even lower concentration might also be acceptable.
Interestingly, the rod filled with this solution contained the least amount of gas after 2 months, but this might be because the particular rod closed better than the others.
The visibility recorded was among the higher candidates, too.
For those reasons, and because the liquid is cheap and easy to produce, it is a promising candidate.

Solutions containing \textit{agar} (\textbf{\#14} and \textbf{\#15}) are even harder to remove from rods if not impossible.
As they might lead to air bubbles, too, which cannot be moved to either side of the rod, agar is not suited for this phantom.

Finally, the synthetic motor oil (\textbf{\#16}) resulted in the highest signal intensity of all candidates.
Besides the question of its toxicity, it seems to be a good alternative to water based liquids.
The silicon oil (\textbf{\#17}) on, the other side, had a low signal compared to most candidates and is therefore not suited.


\section{Distortion}

\subsection{Calculation Methods}
\texttt{DC} and warp values calculated with the simple and iteration method do not differ much.
Moreover, both methods choose very similar thresholds for their calculations (see header of appended '.txt' output file)
As the simple method uses additional information on the rod's true dimension, it is supposed to yield accurate, reliable results.
The iteration process, oblivious to the imaging modality, supports this claim as it produces very similar numbers.


At the moment the iteration method does not take into account the steepness of the DC curve.
In some cases this might result in neglecting the left hand side close to very low percentages.
This happens when the maximum lies roughly in the middle of the current range, but a little bit to the left.
Because the slope is steeper on the left (close to 0\%), a value representing the left hand side (also close to 0\%) yields a much lower result than the value on the right hand side (flat slope).
This should be taken into account during further improvement of the method. \todo{add figure?!}

\subsection{Measured distortion}

Figure \ref{fig:MR_x100_centroids} visualises the centroid shift in significant slices of rod \#5.
It is clearly visible that the shift is bigger at the end of the rod (slice 0) compared to a slice close to its middle (slice 150).
Table \ref{tab:spit-out-5} verifies this.
Furthermore, the distortion measured at the location of the air bubble (slice 304) shows a maximum for the COM shift and a minimum for the DC.
These values originate not from true spatial distortion, but are affected heavily by the presence of air.
In figures \ref{fig:ph2_warpMagnitude_x100} and \ref{fig:ph2_DC_x100} the impact at the area of 110mm to 140mm is clearly visible.
It is imperative for the distortion assessment to use bubble free rods.

\subsection{Irregularities}
\todo{rewrite}
In the case of a slice located at a plastic pane, the CT image will be mostly very bright, as the plastic results in a high intensity.
Consequently, this slice will be marked as irregular as the average brightness differs greatly from the reference slice's.
However, the MRI scan will not show such a increase in brightness at that location, as plastic is not visible for the MRI scanner.
Here the MRI data will not be marked as irregular, provided there is no other cause for a change in brightness, like an air bubble.
Such an absence of liquid would result in a drastic decrease of brightness and result in an irregular slice.
In CT, on the other hand, an air bubble would hardly make a difference, as the surrounding plastic is far brighter than the contained liquid.

\subsection{DC calculation}
\label{sec:discussion_DC}

\subsection{COM calculation}
\todo{rewrite}



%Gnerally this graphs are OK, but we discused before you left that there is need to comment of what we actually see on them, as you explained that to me. Also we dicussed that you could add a figure showing a particular slice on which a certain problem occures (why is the extra-high distortion dettected: a bubble or artefact). 

%We also talked that the XY axes have to properly called, so on X distance from the middle instead of the slice etc. Based on this descriptoion we could write smth clever in the discussion.  

To get a idea of the occurring distortion one should look at both the absolute value of coordinate shift and the dice-coefficient (DC).
An air bubble might lead to a incorrect COM.
Without looking at the dice-coefficient it is hard to tell why this distortion only appears to be present in a few slices.
If both indicators show unexpected local irregularities, a conclusion might be easier to draw.



\todo{discuss what is visible on the graphs in results section}
\todo{interpret data table spit out by script}
\todo{scale of distortion, bigger on the sides as expected}
\todo{only x,y direction. future design needs objects going from left to right, too.}