\chapter*{Preface}

Medicine has always been a scientific area where the crossing of different fields of research accounted for breakthroughs leading to a better understanding and, consequently, improved therapeutic methods.
The success of modern approaches and the development of future techniques owes to the increased interdisciplinary research and work done by both medical personnel such as doctors and nurses, but also scientists like biologists, chemists, physicists and engineers.
As the knowledge about our own human body grows, the problems that we face are becoming more and more complex and need the interdisciplinary expertise.
Even though some therapeutic questions appear easy to answer, because they are easily understood on a general level, the actual treatment of a real patient is something entirely different.
Coming up with a treatment plan becomes exceedingly complex as we try to improve its precision and go towards targeted therapies which are tailor fit to the needs of an individual patient.
Improving their quality of life and chance of survival has always come with increased costs and effort as trade-off.
To make the best treatments available for everyone and, in the long run, also reduce the cost of used resources:
this work shall be a small contribution to this development.