
\chapter{Conclusion and Outlook}
% o Summary of study
%  Most important results and their meaning/impact and conclusions
%  Outlook
%  open questions
%  Next steps
\section{Liquid for phantom}

As topping up over 300 rods regularly is too time consuming, and all water-based liquids continued to evaporate, they are not long-term solution for this phantom.
For short time experiments with this phantom (not airtight rods) \textbf{\#5} is recommended, because the soap allows the air to be moved out of the FOW.
If a new set of (airtight) rods was obtained, adding ascorbic acid to the solution (a combination of \textbf{\#5} and \textbf{\#8}) might be an even better filling:
\begin{itemize}
\item  distilled water
\item  $0.36 \, g/L$ of $NaCl$
\item  $1.96 \, g/L$ of $CuSO_4\cdot5H_2O$
\item  $1 \, g/L$ of soap
\item  $0.36 g/L$ of ascorbic acid
\end{itemize}
In that case, the forming of air bubbles might be either avoided entirely (due to the ascorbic acid) or delayed and then easily taken care of by tilting the whole phantom slightly to move the bubbles out of the FOW.
The long-term behaviour of the mix might lead to adverse properties and should be tested, though.

In the current set-up (not airtight rods), it seems best to fill the rods of the phantom type of oil that does not rot, but yields high signal.
The tested generic motor oil (\textbf{\#16}) fits those requirements.
However, a non toxic alternative would be preferred.


    
\section{Future improvement of software tool}

The developed tool took a simplified approach to the problem by looking only at a single rod.
Future improvements should enable the software to take into account all the rods automatically.
This could be done by a auto-trace function which detects individual rods and applies the already implemented algorithms to each of them separately. 

Distortion assessment can be performed using the generated tool, however, it needs to be implemented on more general scale to be able to measure the distortion for all the rods automatically
