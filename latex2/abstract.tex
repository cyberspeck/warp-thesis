%\chapter*{Kurzzusammenfassung} 

%Purpose, Materials and Methods, Results, Conclusion
%deutsch?\\
\let\oldcleardoublepage\cleardoublepage
\renewcommand\cleardoublepage{}

\chapter*{\abstractname}% Korrekter Name für den Abstract in der jeweiligen Sprache

For radiotherapy treatment planning, knowledge concerning the reliability of the utilised image modality is crucial.
Even though MRI scanners promise superior soft tissue contrast, the geometric precision is not as accurate as CT imaging.
To assess the spatial distortion of a 0.35T MRI scanner, a software tool was developed which uses a custom designed phantom to calculate the occurring deformation and overall shift compared to CT scans.
This was accomplished with use of the freely available python library SimpleITK.
As the implemented tool is not yet able to calculate the distortion throughout the whole field of view, a conclusion whether treatment planning based solely on this scanner would be feasible is not possible at this point.
However, additionally to the development of the script, candidates for a suitable filling of the phantom were produced and tested.
Without a filling, the phantom is made up by a acrylic frame and more than 300 hollow rods.
As the plastic material itself is not visible on MR images, the rods need to be filled with a liquid resulting in acceptable signal intensity.
Synthetic oil was found to yield exceptional signal strength while promising long term reliability.
Water based liquids, on the other hand, pose the issue of evaporation and forming of gas bubbles caused by dissolved gases.
Possible solutions dealing with those problems are not ruled out entirely, especially if thicker rod walls were able to stop the evaporation entirely.
Adding ascorbic acid to a solution of $CuSO_4\cdot5H_2O$ and $NaCl$ might limit the forming of bubbles while soap mobilises them so they can easily be moved to on end of the phantom which is outside of the field of view. 

\let\cleardoublepage\oldcleardoublepage
\newpage
